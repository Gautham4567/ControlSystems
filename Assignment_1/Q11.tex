\documentclass{beamer}

\mode<presentation>
\usepackage{amsmath}
\usepackage{graphicx}
\usepackage{amssymb}
\usepackage{xcolor}
\usepackage{tikz}
\usepackage{schemabloc}
\usetikzlibrary{circuits}
%\usepackage{advdate}
\usepackage{adjustbox}
\usepackage{subcaption}
\usepackage{enumitem}
\usepackage{multicol}
\usepackage{listings}
\usepackage{url}
\def\UrlBreaks{\do\/\do-}
\usetheme{CambridgeUS}
\usecolortheme{rose}
\setbeamertemplate{footline}
{
  \leavevmode%
  \hbox{%
  \begin{beamercolorbox}[wd=\paperwidth,ht=2.25ex,dp=1ex,right]{author in head/foot}%
    \insertframenumber{} / \inserttotalframenumber\hspace*{2ex} 
  \end{beamercolorbox}}%
  \vskip0pt%
}
\setbeamertemplate{navigation symbols}{}

\providecommand{\nCr}[2]{\,^{#1}C_{#2}} % nCr
\providecommand{\nPr}[2]{\,^{#1}P_{#2}} % nPr
\providecommand{\mbf}{\mathbf}
\providecommand{\pr}[1]{\ensuremath{\Pr\left(#1\right)}}
\providecommand{\qfunc}[1]{\ensuremath{Q\left(#1\right)}}
\providecommand{\sbrak}[1]{\ensuremath{{}\left[#1\right]}}
\providecommand{\lsbrak}[1]{\ensuremath{{}\left[#1\right.}}
\providecommand{\rsbrak}[1]{\ensuremath{{}\left.#1\right]}}
\providecommand{\brak}[1]{\ensuremath{\left(#1\right)}}
\providecommand{\lbrak}[1]{\ensuremath{\left(#1\right.}}
\providecommand{\rbrak}[1]{\ensuremath{\left.#1\right)}}
\providecommand{\cbrak}[1]{\ensuremath{\left\{#1\right\}}}
\providecommand{\lcbrak}[1]{\ensuremath{\left\{#1\right.}}
\providecommand{\rcbrak}[1]{\ensuremath{\left.#1\right\}}}
\theoremstyle{remark}
\newtheorem{rem}{Remark}
\newcommand{\sgn}{\mathop{\mathrm{sgn}}}
\providecommand{\abs}[1]{\left\vert#1\right\vert}
\providecommand{\res}[1]{\Res\displaylimits_{#1}} 
\providecommand{\norm}[1]{\lVert#1\rVert}
\providecommand{\mtx}[1]{\mathbf{#1}}
\providecommand{\mean}[1]{E\left[ #1 \right]}
\providecommand{\fourier}{\overset{\mathcal{F}}{ \rightleftharpoons}}
%\providecommand{\hilbert}{\overset{\mathcal{H}}{ \rightleftharpoons}}
\providecommand{\system}{\overset{\mathcal{H}}{ \longleftrightarrow}}
	%\newcommand{\solution}[2]{\textbf{Solution:}{#1}}
%\newcommand{\solution}{\noindent \textbf{Solution: }}
\providecommand{\dec}[2]{\ensuremath{\overset{#1}{\underset{#2}{\gtrless}}}}
\newcommand{\myvec}[1]{\ensuremath{\begin{pmatrix}#1\end{pmatrix}}}
\let\vec\mathbf

\lstset{
%language=C,
frame=single, 
breaklines=true,
columns=fullflexible
}

\numberwithin{equation}{section}

\title{Control Systems Problem 11}
\author{Gautham Yadagiri\\ EE19BTECH11008\\IIT Hyderabad.}

\date{\today} 
\begin{document}

\begin{frame}
\titlepage
\end{frame}

\section*{Content}
\begin{frame}
\tableofcontents
\end{frame}
\section{Problem}
\begin{frame}
\frametitle{Problem}
A system is described by the following differential
equation

\begin{equation}
\frac{d^2x}{dt^2}+2\frac{dx}{dt}+3x=1
\end{equation}

with the initial conditions x(0) = 1, x'(0) = -1.
Show a block diagram of the system, giving its
transfer function and all pertinent inputs and outputs.
(Hint: the initial conditions will show up as
added inputs to an effective system with zero initial
conditions.)
\end{frame}


\section{Solution}
\begin{frame}
\frametitle{Solution}
Consider the following equation
\begin{equation}
\frac{d^2x}{dt^2}+2\frac{dx}{dt}+3x=r(t)
\end{equation}
where r(t)=1.
\vspace{2mm}

On applying laplace transform on both sides we get
\begin{align}
s^2X(s)-s+1+2sX(s)-2+3X(s) = R(s)
\nonumber\\
(s^2+2s+3)X(s)-s-1 = R(s)
\nonumber\\
X(s) = \frac{R(s)}{(s^2+2s+3)}+\frac{s+1}{(s^2+2s+3)}
\end{align}

where R(s) = $\frac{1}{s}$

\end{frame}

\section{Block Diagram}
\begin{frame}
\frametitle{Block Diagram}
\begin{equation}
    X(s) = \frac{R(s)}{(s^2+2s+3)}+\frac{s+1}{(s^2+2s+3)}
\end{equation}
Here, s+1 is due to the initial conditions and are getting added to the input R(s). So, the block diagram is 

\vspace{5mm}

\begin{tikzpicture}[block/.style={rectangle, draw=black!80, very thick, text width=30mm, minimum height=15mm}]

    \node[block](TF){\hspace{7mm}$\frac{1}{(s^2+2s+3)}$};
    \sbCompSum[-12]{a}{}{}{+}{+}{}
    \draw[->][black, thick] (-7,0)--node[above] {R(s)}(a);
    \draw[->][black, thick] (a)--(TF);
    \draw[->][black, thick] (-4.6,-2)--node[right] {s+1}(a);
    \draw[->][black, thick] (TF)--node[above] {X(s)}(4,0);
\end{tikzpicture}

\end{frame}

\section{Plots}
\begin{frame}
\frametitle{Plots}
\begin{figure}[ht]
    \centering
    \includegraphics[width=0.7\columnwidth]{Bode.png}
    \caption{Bode Plot of Transfer Function}
    \label{fig:my_label}
\end{figure}
\end{frame}

\begin{frame}
\frametitle{Plots}
\begin{figure}[ht]
    \centering
    \includegraphics[width=0.7\columnwidth]{Output.png}
    \caption{Output Response}
    \label{fig:my_label}
\end{figure}
\end{frame}



\end{document}
